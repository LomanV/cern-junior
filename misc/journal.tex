\documentclass[12pt, letterpaper, twoside]{article}
\usepackage[top=2cm, bottom=2cm, left=1.5cm, right=1.5cm]{geometry}
\usepackage[francais]{babel}
\usepackage{array}
\usepackage[T1]{fontenc}

\usepackage[sf]{titlesec}
\renewcommand{\familydefault}{\sfdefault}

\usepackage{enumitem,amssymb}
\newlist{todolist}{itemize}{2}
\setlist[todolist]{label=$\square$}
\usepackage{pifont}
\newcommand{\cmark}{\ding{51}}%
\newcommand{\xmark}{\ding{55}}%
\newcommand{\done}{\rlap{$\square$}{\raisebox{2pt}{\large\hspace{1pt}\cmark}}%
\hspace{-2.5pt}}
\newcommand{\wontfix}{\rlap{$\square$}{\large\hspace{1pt}\xmark}}

\begin{document}
\section{Progression} 

\begin{itemize}
\item Tâches à effectuer: 
\begin{todolist}
	\item[\done] Créer le JOURNAL
	\item[\done] Lire complètement le descriptif général 
	\item[\done] S'inscrire en binôme 
	\item[\done] Makefile  
	\item[\done] Vecteur finie (pleinement opérationnelle et testée)
	\item[\done] Fichier REPONSES
	\item[\done] Oscillateur
	\item[\done] Intégrateur d'Euler-Cromer
	\item[\done] Pendule
	\item[\done] Ressort
	\item[\done] Systeme
	\item[\done] ExerciceP10
	\item[\done] Graphisme: cadre général 
	\item[\done] Graphisme: Pendule
	\item[\done] OscillateurCouple 
	\item[\done] PenduleDouble
	\item[\done] Espace des phases
	\item[\done] Integrateur Newmark
	\item[\done] Fichier CONCEPTION
	\item[\done] Fichier README
	\item[\done] Fichier NOMS
\end{todolist}
\end{itemize}

\rule{\textwidth}{0.4pt}

\section{Suivi:}

\subsection{Semaine 1:}
\begin{itemize}
	\item Création du module Vector3D
	\item Testé et quasiment finalisé
	\item Mise en place du répertoire GitHub 
\end{itemize}

\rule{\textwidth}{0.4pt}

\subsection{Semaine 2:}
\begin{itemize}
\item Mise en place de l'environnement QT et réalisation du tutoriel graphisme P12 

\item Prise en main plus poussée de l'environnement graphique, implémentation de plusieurs exemples simples (polygônes plans, cubes, sphères, sphères en mouvement).

\item Création du répertoire Github qui sera, à terme, le répertoire du projet final.

\item Résolution d'un bug très pénible dû à QMake sur Macintosh qui rendait la compilation de librairie impossible (ouf!).

\item Création du module `physics' qui contenant la classe Particle.

\item Conception d'une toute première (et fort rudimentaire) simulation physique en temps réel à sortie graphique: un simulateur de problème à $n$ corps (Calculs par somme direct, donc $\mathcal{O}(n^2)$. Devient très couteux à partir de $\sim$1000 particules.)

\item Modification des opérateurs de calcul algébrique sur les vecteurs et surcharge des opérateurs d'affichage

\item Ajout de la méthode rotate à la classe Vector3D
\item Ajout de la méthode force magnétique à la classe Particule
\item Création d'un fichier test pour la classe Particule, ce dernier écrit en sortie sur un fichier au format txt
\end{itemize}

\rule{\textwidth}{0.4pt}

\subsection{Semaine 3:}
\begin{itemize}
\item Cela sort un peu du cadre du projet mais je décide d'essayer d'implémenter l'algorithme de Barnes-Hut pour accélérer les calculs des interactions gravitationnelles des particules. Celui-ci est en  $\mathcal{O}(n\log{}n)$ donc très intéressant si on a un grand nombre de particules en jeu.

\item Après avoir résolu un bug fort pénible, l'implémentation de Barnes-Hut est un succès. La simulation de problème à $n$ corps pour $10^4$ voire $10^5$ particules peut maintenant être exécutée aisément même sur un ordinateur peu puissant. Évidemment l'algorithme peut facilement être adapté pour le calcul de forces électromagnétique.

\item Finalisation de la classe Particle et écriture d'un fichier test. Modification du fichier test des vecteurs, on préfère qu'il écrive en sortie sur une fenêtre terminal
\end{itemize}

\rule{\textwidth}{0.4pt}

\subsection{Semaine 4 :}

\begin{itemize}
\item Première implémentation des éléments, une classe abstraire représente les éléments en général, de cette classe héritent deux sous classes pour les éléments droits et courbes.

\item Nous modifions par la suite notre conception des éléments afin d'éviter une duplication de code. 

\item Conception de la classe accélérateur et premier fichier test, erroné

\item Correction du fichier test
\end{itemize}

\rule{\textwidth}{0.4pt}

\subsection{Semaine 5 :}

\begin{itemize}
\item Complétion de la classe élément
\item Premiers dessins de cylindres et sections de tore sur l'environnement QT
\item Ajout d'un classe pour la gestion des couleurs afin d'alléger le code
\end{itemize}

\rule{\textwidth}{0.4pt}

\subsection{Semaine 6 :}
\begin{itemize}
\item Finalisation des dessins de cylindres et sections de tore, adaptation des méthodes au cas de notre projet
\item Affichage des premiers éléments
\item Révision de la conception des méthodes afin de rendre plus naturelle et pratique la construction graphique des éléments.
\item Gestion des exceptions
\item Ajout de namespace pour une meilleure gestion des exceptions ainsi que des constantes physiques.
\end{itemize}

\rule{\textwidth}{0.4pt}

\subsection{Semaine 7 :}
\begin{itemize}
\item Gestion des déplacements à la souris
\item Ajout de la classe Faisceau
\item Révision complète des anciennes classes pour implémenter les faisceaux 
\end{itemize}

\rule{\textwidth}{0.4pt}

\subsection{Semaine 8 :}
\begin{itemize}
\item Révision de la classe Particule, ajout d'une classe Point Charge pour rendre le code plus léger et intuitif
\item Ajout de la coordonnée curviligne de l'accélérateur (ainsi que propre à chaque élément)
\item Utilisation de cette dernière pour la conception des faisceaux circulaires
\end{itemize}

\rule{\textwidth}{0.4pt}

\subsection{Semaine 9 :}

\rule{\textwidth}{0.4pt}

\subsection{Semaine 10 :}
\begin{itemize}
\item Dernière révision de l'affichage graphique, rendu plus épuré
\item Ajout de différents points de vue afin de mieux suivre l'évolution des particules, révision complète de la gestion des évènements clavier.
\end{itemize}

\rule{\textwidth}{0.4pt}

\subsection{Semaine 11 :}

\begin{itemize}
\item Début de l'implémentation des interractions interparticulaires à l'aide de l'algorithme de Barnes-Hut implémenté en semaine 3. Test avec $7500$ particules. 
\item Ajout du mode matrice affichant les boites de l'algotithme (touche `m')
\end{itemize}

\rule{\textwidth}{0.4pt}

\subsection{Semaine 12 :}

\rule{\textwidth}{0.4pt}

\subsection{Semaine 13 :}

\rule{\textwidth}{0.4pt}

\end{document}
