\documentclass[12pt, letterpaper, twoside]{article}
\usepackage[top=3cm, bottom=3cm, left=3cm, right=3cm]{geometry}
\usepackage{amsmath}
\usepackage{amssymb}

\usepackage[french]{babel}
\usepackage[utf8]{inputenc}

\title{Réponses}
\author{Vock Raphaël, Vezin Lomàn}

\begin{document}
Nous avons fait le choix d'une représentation cartésienne des vecteurs de \textbf{R}$^3$, ces derniers sont donc modélisés par une classe \texttt{Vector3D} dont les trois attributs x, y, z sont des \texttt{double}. L'accès aux attributs est privé, et se fait par le bisais d'un getter qui les retourne dans un \texttt{array} de dimension 3. Nous utilisons une méthode \texttt{is\_zero} qui renvoie \texttt{true} si la norme carrée du vecteur en instance est inférieure à une marge \texttt{DEFAULT\_EPSILON} donnée. Les attributs peuvent toutefois être initialisés par le biais s'un setter. La classe dispose de plus d'un constructeur faisant également office de constructeur par défaut, initialisant en l'absence d'argument chacune des coordonnées à 0.
\end{document}
