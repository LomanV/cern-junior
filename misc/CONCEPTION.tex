\documentclass[12pt, letterpaper, twoside]{article}
\usepackage[top=3cm, bottom=3cm, left=3cm, right=3cm]{geometry}
\usepackage[fleqn]{amsmath}
\usepackage{amssymb}
\usepackage{tikz}
\usetikzlibrary{arrows}
\usetikzlibrary{shapes,decorations}
\usepackage{comment}

\usepackage[utf8]{inputenc}
\usepackage[T1]{fontenc}
\usepackage[french]{babel}

\title{Projet: conception}
\author{Lomàn Vezin \\ Raphaël Vock}

\newcommand{\T}[1]{\texttt{#1}}
\newcommand{\V}[0]{\texttt{Vector3D}}
\newcommand{\double}{\texttt{double}}

\usepackage{hyperref}

\usepackage[utf8]{inputenc}
\usepackage[T1]{fontenc}
\usepackage{microtype}
\DisableLigatures[>]{encoding = T1}

\def\O{\mathcal{O}}
\def\epsilon{\varepsilon}

\renewcommand{\familydefault}{\sfdefault}
\begin{document}
\maketitle
\section{Représentation schématique des hiérarchies}
Ci-dessous deux schémas représentant les hiérarchies relationnelles des classes.

\subsection{Légende}
\begin{tikzpicture}[transform shape,mylabel/.style={thin, draw=black, align=center, minimum width=0.5cm, minimum height=0.5cm,fill=white,font=\normalsize}]
\node (L0) at (0,0) {};
\node (R0) at (3.5,0) {};
\node (T0) at (3.5,0) [label=right: hérite de] {};

\node (L1) at (0,-1) {};
\node (R1) at (3.5,-1) {};
\node (T1) at (3.5,-1) [label=right: hérite de \textit{via} \T{template<>}] {};

\node (L2) at (0,-2) {};
\node (R2) at (3.5,-2) {};
\node (T2) at (3.5,-2) [label=right: contient une ou plusieurs instances de] {};

\node (L3) at (0,-3) {};
\node (R3) at (3.5,-3) {};
\node (T3) at (3.5,-3) [label=right: contient un ou plusieurs pointeurs vers des instances de] {};

\draw
(L0) edge [->] (R0)
(L1) edge [->] node[mylabel] {\T{template<>}} (R1)
(L2) edge [->, red] (R2)
(L3) edge [->, red, dashed] (R3);

\end{tikzpicture}

\subsection{Des classes liées aux objets de la simulation}
\begin{tikzpicture}[transform shape,mylabel/.style={thin, draw=black, align=center, minimum width=0.5cm, minimum height=0.5cm,fill=white,font=\normalsize}]

\node (P0) at (2,0) {\T{Vector3D}};
\node (P1) at (2,-3) {\T{PointCharge}}; 
\node (P9) at (2, -6) {\T{Particle}};
\node (P3) at (12, -10.5) {\T{Beam}};
\node (P4) at (12, 0) {\T{Accelerator}};
\node (P6) at (12, -4) {\T{Element}};
\node (P7) at (12, -7.32) {\T{Node}};
\node (P8) at (7.5, -7.32) {\T{Box}};

\draw
(P4) edge [->] node[mylabel] {\T{vector<unique\_ptr<>\,>}} (P6)
(P3) edge [-> , bend left = 20] node[mylabel] {\T{std::vector<unique\_ptr<unique\_ptr<>\,>\,>}} (P9.south)

(P7) edge [-> , loop right, dashed, red] (P7.south)
(P6) edge [-> , loop right, dashed, red] (P6.south)
(P4) edge [-> , red] (P0)
(P1) edge [->] (P0)
(P9) edge [->] (P1)
(P6) edge [->] (P7)
(P7) edge [-> , red] (P8)
(P4) edge [-> , dashed, red] (P9)
(P6.west) edge [-> , red] (P0)
(P7) edge [-> , red] (P1.south east)
(P8) edge [-> , red] (P0)
(P9) edge [-> , bend left = 60, red] (P0)
(P4.south east) edge [-> , bend left = 60, dashed, red] (P3.north)
(P3) edge [-> , bend left = -60, dashed, red] (P4.east);

\end{tikzpicture}

\subsection{Des classes liées à la représentation de la simulation}

\begin{tikzpicture}
\node (P0) at (4.5,0) {\T{Drawable}};
\node (P1) at (13.5,0) {\T{Canvas}};
\node (P2) at (9,-2) {\T{Element}};
\node (P3) at (12, -2) {\T{TextView}};
\node (P4) at (15, -2) {\T{OpenGLView}};
\node (P5) at (6,-2) {\T{Accelerator}};
\node (P6) at (3,-2) {\T{Particle}};
\node (P7) at (0,-2) {\T{Box}};
\node (P8) at (10.5,-7) {\T{AcceleratorWidgetConsole}};
\node (P9) at (10.5,-4.5) {\T{AcceleratorWidgetGL}};

\path [->]
(P0) edge [dashed, red] (P1)
(P2) edge (P0)
(P5) edge (P0)
(P6) edge (P0)
(P7) edge (P0)
(P3) edge (P1)
(P4) edge (P1)
(P8) edge (P4)
(P9) edge (P3)
(P8) edge (P5)
(P9) edge (P5.south east);
\end{tikzpicture}

\ \linebreak
\ \linebreak
\section{Quelques éclaircissements}


\end{document}
